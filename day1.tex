% ! TEX program = xelatex
% template author: trswnca@yeah.net
% issues on github.com/trswnca/noip-latex
%%%%%%%%%%%%%%%%%%%%%%%%%%%%% Define Article %%%%%%%%%%%%%%%%%%%%%%%%%%%%%%%%%%
\documentclass{article}
%%%%%%%%%%%%%%%%%%%%%%%%%%%%%%%%%%%%%%%%%%%%%%%%%%%%%%%%%%%%%%%%%%%%%%%%%%%%%%%

%%%%%%%%%%%%%%%%%%%%%%%%%%%%% Using Packages %%%%%%%%%%%%%%%%%%%%%%%%%%%%%%%%%%
\usepackage{geometry}
\usepackage{graphicx}
\usepackage{amssymb}
\usepackage{amsmath}
\usepackage{amsthm}
\usepackage{empheq}
\usepackage{mdframed}
\usepackage{booktabs}
\usepackage{lipsum}
\usepackage{graphicx}
\usepackage{color}
\usepackage{psfrag}
\usepackage{pgfplots}
\usepackage{bm}
\usepackage{xeCJK}
\usepackage{tabularx}
\usepackage{xeCJKfntef}
\usepackage{listings}
%%%%%%%%%%%%%%%%%%%%%%%%%%%%%%%%%%%%%%%%%%%%%%%%%%%%%%%%%%%%%%%%%%%%%%%%%%%%%%%

% Other Settings

%%%%%%%%%%%%%%%%%%%%%%%%%% Page Setting %%%%%%%%%%%%%%%%%%%%%%%%%%%%%%%%%%%%%%%
\geometry{a4paper}

%%%%%%%%%%%%%%%%%%%%%%%%%% Define some useful colors %%%%%%%%%%%%%%%%%%%%%%%%%%
\definecolor{ocre}{RGB}{243,102,25}
\definecolor{mygray}{RGB}{243,243,244}
\definecolor{deepGreen}{RGB}{26,111,0}
\definecolor{shallowGreen}{RGB}{235,255,255}
\definecolor{deepBlue}{RGB}{61,124,222}
\definecolor{shallowBlue}{RGB}{235,249,255}
%%%%%%%%%%%%%%%%%%%%%%%%%%%%%%%%%%%%%%%%%%%%%%%%%%%%%%%%%%%%%%%%%%%%%%%%%%%%%%%

%%%%%%%%%%%%%%%%%%%%%%%%%% Define an orangebox command %%%%%%%%%%%%%%%%%%%%%%%%
\newcommand\orangebox[1]{\fcolorbox{ocre}{mygray}{\hspace{1em}#1\hspace{1em}}}
%%%%%%%%%%%%%%%%%%%%%%%%%%%%%%%%%%%%%%%%%%%%%%%%%%%%%%%%%%%%%%%%%%%%%%%%%%%%%%%

%%%%%%%%%%%%%%%%%%%%%%%%%%%% English Environments %%%%%%%%%%%%%%%%%%%%%%%%%%%%%
\newtheoremstyle{mytheoremstyle}{3pt}{3pt}{\normalfont}{0cm}{\rmfamily\bfseries}{}{1em}{{\color{black}\thmname{#1}~\thmnumber{#2}}\thmnote{\,--\,#3}}
\newtheoremstyle{myproblemstyle}{3pt}{3pt}{\normalfont}{0cm}{\rmfamily\bfseries}{}{1em}{{\color{black}\thmname{#1}~\thmnumber{#2}}\thmnote{\,--\,#3}}
\theoremstyle{mytheoremstyle}
\newmdtheoremenv[linewidth=1pt,backgroundcolor=shallowGreen,linecolor=deepGreen,leftmargin=0pt,innerleftmargin=20pt,innerrightmargin=20pt,]{theorem}{Theorem}[section]
\theoremstyle{mytheoremstyle}
\newmdtheoremenv[linewidth=1pt,backgroundcolor=shallowBlue,linecolor=deepBlue,leftmargin=0pt,innerleftmargin=20pt,innerrightmargin=20pt,]{definition}{Definition}[section]
\theoremstyle{myproblemstyle}
\newmdtheoremenv[linecolor=black,leftmargin=0pt,innerleftmargin=10pt,innerrightmargin=10pt,]{problem}{Problem}[section]
%%%%%%%%%%%%%%%%%%%%%%%%%%%%%%%%%%%%%%%%%%%%%%%%%%%%%%%%%%%%%%%%%%%%%%%%%%%%%%%

%%%%%%%%%%%%%%%%%%%%%%%%%%%%%%% Plotting Settings %%%%%%%%%%%%%%%%%%%%%%%%%%%%%
\usepgfplotslibrary{colorbrewer}
\pgfplotsset{width=8cm,compat=1.9}
%%%%%%%%%%%%%%%%%%%%%%%%%%%%%%%%%%%%%%%%%%%%%%%%%%%%%%%%%%%%%%%%%%%%%%%%%%%%%%%

\lstdefinestyle{mystyle}{
  basicstyle=%
    \ttfamily
    \lst@ifdisplaystyle\small\fi
}

\lstset{basicstyle=\ttfamily,style=mystyle,breaklines=true,upquote=true}

\definecolor{lightgrey}{rgb}{0.9,0.9,0.9}
\definecolor{frenchplum}{RGB}{190,20,83}
\definecolor{CPPLight}  {HTML} {686868}
\definecolor{CPPSteel}  {HTML} {888888}
\definecolor{CPPDark}   {HTML} {262626}
\definecolor{CPPBlue}   {HTML} {4172A3}
\definecolor{CPPGreen}  {HTML} {487818}
\definecolor{CPPBrown}  {HTML} {A07040}
\definecolor{CPPRed}    {HTML} {AD4D3A}
\definecolor{CPPViolet} {HTML} {7040A0}
\definecolor{CPPGray}  {HTML} {B8B8B8}
\definecolor{mygreen}{RGB}{34,139,34}
\definecolor{mygray}{rgb}{0.5,0.5,0.5}
\definecolor{mymauve}{rgb}{0.58,0,0.82}
\definecolor{myred}{RGB}{178,34,34}
\lstset{
    tabsize = 2,
    basicstyle = \small\ttfamily, 
    columns=fullflexible,       
    numbers=left,
    frame=l,
    breaklines=true, 
    backgroundcolor=\color{white},
    keywordstyle=\color{mygreen},
    numberstyle=\footnotesize\color{darkgray},
    commentstyle=\color{mygray},
    stringstyle=\color{myred}\ttfamily,
    showstringspaces=false,
    escapeinside={\%*}{*)},
    captionpos=t, 
    rulesepcolor=\color{red!20!green!20!blue!20},
    language=c++,
    xleftmargin=0.5cm,
    morekeywords={alignas,continute,friend,register,true,alignof,decltype,goto,
    reinterpret_cast,try,asm,defult,if,return,typedef,auto,delete,inline,short,
    typeid,bool,do,int,signed,typename,break,double,long,sizeof,union,case,
    dynamic_cast,mutable,static,unsigned,catch,else,namespace,static_assert,using,
    char,enum,new,static_cast,virtual,char16_t,char32_t,explict,noexcept,struct,
    void,export,nullptr,switch,volatile,class,extern,operator,template,wchar_t,
    const,false,private,this,while,constexpr,float,protected,thread_local,
    const_cast,for,public,throw,std},
}

\begin{document}

  \begin{center}
    \huge{\textbf{第40届全国青少年信息学奥林匹克竞赛}}
    ~\\[10pt]
    \huge{CCF NOIP 2021}
    ~\\[10pt]
    \huge{模拟赛\ 第一试}
    ~\\[10pt]
    \Large{时间:08:30 $\sim$ 13:00}
  \end{center}    

  \begin{table}[htbp]
      \label{tab:problems}
      \begin{center}
        \begin{tabular}{|p{2.5cm}|p{2.5cm}|p{2.5cm}|p{2.5cm}|p{2.5cm}|}
          \hline
          题目名称 & A & B & C & D\\ \hline
          题目类型 & 传统型 & 传统型 & 传统型 & 传统型\\ \hline
          目录    & pa & pb & pc & pd \\ \hline
          可执行文件名 & pa & pb & pc & pd \\ \hline
          输入文件名 & pa.in & pb.in & pc.in & pd.in \\ \hline
          输出文件名 & pa.out & pb.out & pc.out & pd.out \\ \hline
          每个测试点时限 & 1.5秒 & 1.0秒 & 1.0秒 & 1.0秒 \\ \hline
          内存限制 & 1GB & 1GB & 1GB & 1GB \\ \hline
          子任务数目 & 20 & 20 & 20 & 20 \\ \hline
          测试点是否等分 & 是 & 是 & 是 & 是 \\ \hline
          
        \end{tabular}
      \end{center}
  \end{table}

      提交源程序文件名
  
  \begin{table}[htbp]
    \label{tab:subname}
    \begin{center}
        \begin{tabular}{|p{2.5cm}|p{2.5cm}|p{2.5cm}|p{2.5cm}|p{2.5cm}|}
          \hline
          对于C++语言 & pa.cpp & pb.cpp & pc.cpp & pd.cpp \\ \hline

      \end{tabular}
    \end{center}
  \end{table}

  编译命令

  \begin{table}[htbp]
    \label{tab:lang}
    \begin{center}
        \begin{tabular}{|p{2.5cm}|p{11.3 cm}|}
          \hline
          对于C++语言 & g++ \%s.cpp -o \%s\\ \hline

      \end{tabular}
    \end{center}
  \end{table}

  \textbf{【注意事项(请仔细阅读)】}

  \begin{enumerate}
    \item 选手提交的源程序必须存放在\CJKunderdot{已建立}好的,且 \CJKunderdot{带有样例文件和下发文件}的文件夹 中,文件夹名称与对应试题英文名一致。
    \item 文件名(包括程序名和输入输出文件名)必须使用英文小写。
    \item C++ 中函数 main() 的返回值类型必须是 int,值必须为 0。
    \item \CJKunderdot{对于因未遵守以上规则对成绩造成的影响,相关申诉不予受理}。
    \item 若无特殊说明,结果比较方式为\CJKunderdot{忽略行末空格、文末回车后的全文比较}。
    \item 程序可使用的栈空间大小与该题内存空间限制一致。
    \item 在终端中执行命令 ulimit -s unlimited 可将当前终端下的栈空间限制放大,但你使用的栈空间大小不应超过题目限制。
    \item 每道题目所提交的\CJKunderdot{代码文件大小限制为} 100KB。
    \item 若无特殊说明,输入文件与输出文件中同一行的相邻整数均使用一个空格分隔。
    \item 输入文件中可能存在行末空格,请选手使用更完善的读入方式(例如 scanf 函数) 避免出错。
    \item 直接复制 PDF 题面中的多行样例,数据将带有行号,建议选手直接使用对应目录下的样例文件进行测试。
    \item 使用 std::deque 等 STL 容器时,请注意其内存空间消耗。
    \item 请务必使用题面中规定的的编译参数,保证你的程序在本机能够通过编译。此外\CJKunderdot{不允许在程序中手动开启其他编译选项},一经发现,本题成绩以 0 分处理
  \end{enumerate}

  \clearpage

  \begin{center}
    \huge{A}
  \end{center}
  ~\\[10pt]

  \textbf{【题目描述】}

  CYCが大好きです

  \textbf{【输入格式】}

  从文件 \textbf{\textit{pa.in}} 中读入数据。

  \textbf{【输出格式】}

  输出到文件 \textbf{\textit{pa.out}} 中。

  \textbf{【样例 1 输入】}

  \textbf{【样例 1 输出】}

  \textbf{【样例 1 解释】}
  

\end{document}